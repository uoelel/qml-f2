% Options for packages loaded elsewhere
\PassOptionsToPackage{unicode}{hyperref}
\PassOptionsToPackage{hyphens}{url}
%
\documentclass[
]{article}
\usepackage{amsmath,amssymb}
\usepackage{lmodern}
\usepackage{iftex}
\ifPDFTeX
  \usepackage[T1]{fontenc}
  \usepackage[utf8]{inputenc}
  \usepackage{textcomp} % provide euro and other symbols
\else % if luatex or xetex
       %%% MODIFIED: unicode-math conflics with expex; mathspects conflicts with glossaries/leipzig
  \usepackage{fontspec} % \usepackage{unicode-math}
  \defaultfontfeatures{Scale=MatchLowercase}
  \defaultfontfeatures[\rmfamily]{Ligatures=TeX,Scale=1}
  \setmainfont[]{DejaVu Sans}
\fi
% Use upquote if available, for straight quotes in verbatim environments
\IfFileExists{upquote.sty}{\usepackage{upquote}}{}
\IfFileExists{microtype.sty}{% use microtype if available
  \usepackage[]{microtype}
  \UseMicrotypeSet[protrusion]{basicmath} % disable protrusion for tt fonts
}{}
\makeatletter
\@ifundefined{KOMAClassName}{% if non-KOMA class
  \IfFileExists{parskip.sty}{%
    \usepackage{parskip}
  }{% else
    \setlength{\parindent}{0pt}
    \setlength{\parskip}{6pt plus 2pt minus 1pt}}
}{% if KOMA class
  \KOMAoptions{parskip=half}}
\makeatother
\usepackage{xcolor}
\IfFileExists{xurl.sty}{\usepackage{xurl}}{} % add URL line breaks if available
\IfFileExists{bookmark.sty}{\usepackage{bookmark}}{\usepackage{hyperref}}
\hypersetup{
  pdftitle={QML - Formative Assessment 2},
  pdfauthor={YOUR EXAM NUMBER},
  hidelinks,
  pdfcreator={LaTeX via pandoc}}
\urlstyle{same} % disable monospaced font for URLs
\usepackage[margin=1in]{geometry}
\usepackage{graphicx}
\makeatletter
\def\maxwidth{\ifdim\Gin@nat@width>\linewidth\linewidth\else\Gin@nat@width\fi}
\def\maxheight{\ifdim\Gin@nat@height>\textheight\textheight\else\Gin@nat@height\fi}
\makeatother
% Scale images if necessary, so that they will not overflow the page
% margins by default, and it is still possible to overwrite the defaults
% using explicit options in \includegraphics[width, height, ...]{}
\setkeys{Gin}{width=\maxwidth,height=\maxheight,keepaspectratio}
% Set default figure placement to htbp
\makeatletter
\def\fps@figure{htbp}
\makeatother
\setlength{\emergencystretch}{3em} % prevent overfull lines
\providecommand{\tightlist}{%
  \setlength{\itemsep}{0pt}\setlength{\parskip}{0pt}}
\setcounter{secnumdepth}{5}
\ifLuaTeX
  \usepackage{selnolig}  % disable illegal ligatures
\fi

\title{QML - Formative Assessment 2}
\author{YOUR EXAM NUMBER}
\date{2023-11-09}

\begin{document}
\maketitle

\section{Overview}\label{overview}

\textbf{PLEASE READ CAREFULLY}

\textbf{DUE Week 10 - Thu 23 November at noon}

You must include your \textbf{exam number as the author} in the document
preamble above.

You'll notice that line 10 of the preamble says
\texttt{mainfont:\ DejaVu\ Sans}. We would appreciate it if you'd use
this font (or at least some other sans-serif font), but it's fine to use
any other sans-serif font, like Arial. \textbf{Try to render this Rmd
file now, before making any changes, to see if you have this font
installed.}

\begin{itemize}
\tightlist
\item
  If you get an error message saying that DejaVu Sans could not be
  found, you can download it here:
  \url{http://sourceforge.net/projects/dejavu/files/dejavu/2.37/dejavu-fonts-ttf-2.37.zip}.
\item
  Then, install it appropriately for your operating system.

  \begin{itemize}
  \tightlist
  \item
    Here's a guide for Windows:
    \url{https://www.digitaltrends.com/computing/how-to-install-fonts-in-windows-10/}.
  \item
    Here's a guide for Mac:
    \url{https://support.apple.com/en-gb/HT201749}.
  \end{itemize}
\item
  If you are having issues with this, feel free to use any other
  sans-serif font you have installed on your machine, like Arial.
\end{itemize}

This assessment covers \textbf{Weeks 1 to 8}.

You are asked to \textbf{read, wrangle, summarise, plot and model data}
from Lorson, Alex, Cummins, Chris \& Rohde, Hannah. 2021. `Strategic use
of (un)certainty expressions'. Frontiers in Communication. 6. DOI:
\url{https://doi.org/10.3389/fcomm.2021.635156}.

Read the paper abstract to get a sense of what the study is about.

You will analyse the data from the second study in the paper. You can
skim Section 3.2 of the paper for an overview of the study. You will
work on the \texttt{Participant}, \texttt{Response}, \texttt{Scenario},
\texttt{Gender} and \texttt{Age} columns.

\textbf{When you are ready to submit}:

\begin{enumerate}
\def\labelenumi{\arabic{enumi}.}
\tightlist
\item
  Render the Rmd file to \textbf{PDF}.
\item
  \textbf{Rename} the PDF to your exam number only.
\item
  \textbf{Upload} the PDF file to Learn.
\end{enumerate}

\section{Read the data}\label{read-the-data}

\textbf{Instructions}:

\begin{itemize}
\tightlist
\item
  Read the data in \texttt{data/lorson2021-Study2d.csv}.
\item
  Write a paragraph describing the data frame. Include the following
  information (some of these will require you to write R code):

  \begin{itemize}
  \tightlist
  \item
    Number of observations and columns.
  \item
    Number of participants (\texttt{Participant}).
  \item
    Minimum and maximum number of observations per participant.
  \item
    Levels of \texttt{Response}.
  \item
    Levels of \texttt{Gender}.
  \item
    Levels of \texttt{Scenario}.
  \end{itemize}
\end{itemize}

\section{Calculate summary measures}\label{calculate-summary-measures}

\textbf{Instructions}:

\begin{itemize}
\tightlist
\item
  Since all the relevant variables except \texttt{Age} are categorical,
  it might be useful to just report for these the number of each
  response (``know'' vs ``believe'') in the different combinations of
  \texttt{Scenario} and \texttt{Gender}.
\item
  Write R code to obtain those counts. If you want you can write a
  markdown table with the counts. You can use this online tool to create
  and copy-paste a markdown table:
  \url{https://www.tablesgenerator.com/markdown_tables}.
\item
  Although we will not include \texttt{Age} in our model later, it is
  customary to provide readers with general socio-demographic
  information on the participants and this usually includes summary
  measures for the participants' age.
\item
  Go ahead and calculate the appropriate summary measures for
  \texttt{Age}. Write a sentence reporting the measures.
\end{itemize}

\section{Plot the data}\label{plot-the-data}

\textbf{Instructions}:

\begin{itemize}
\tightlist
\item
  Now plot the data as you see fit. Remember that the focus of the study
  is how participants use ``know'' vs ``believe'' (\texttt{Response}) so
  you might want your plots to focus on that variable, but feel free to
  also plot combinations of any of \texttt{Participant},
  \texttt{Response}, \texttt{Scenario}, \texttt{Gender} and
  \texttt{Age}.
\item
  For each plot, write a brief description (include information on the
  components of the plot like axes, colour, panels etc and briefly
  describe the patterns you see).
\item
  Plotting the data is a good way of familiarising yourself with it, so
  that when it comes to modelling you have a better sense of what the
  data looks like!
\end{itemize}

\section{Model the data}\label{model-the-data}

\textbf{Instructions}

\begin{itemize}
\tightlist
\item
  Finally, you can model the data and write a model report.
\item
  You are trying to answer the following research question: \emph{Do
  participants show a preference for ``know'' vs ``believe'' depending
  on their gender and the scenario they are in?}.
\item
  Follow these steps to help you model the data:

  \begin{itemize}
  \tightlist
  \item
    Identify the outcome variable and the predictors.
  \item
    Do you need to transform the variables and/or reorder levels?
  \item
    Identify the appropriate distribution for the outcome variable.
  \item
    Is an interaction term necessary?
  \end{itemize}
\end{itemize}

\section{Results}\label{results}

\textbf{Instructions}

\begin{itemize}
\tightlist
\item
  You should include at least two paragraphs, (a) one reporting the
  model specifications (outcome, distribution of the outcome,
  predictors, coding, \ldots) and (b) one reporting the results from the
  model (estimates and estimates' SD, Credible Intervals, conditional
  posteriors and differences across different levels, etc\ldots)
\item
  Include plots of the posterior distributions of each coefficient
  (listed in the model summary) and the conditional posterior
  distributions. You can also include the posterior distributions of
  differences among levels not covered by the coefficients, if any.
\end{itemize}

We fitted a Bayesian model\ldots{}

According to the results, \ldots{}

\section{Discuss the results}\label{discuss-the-results}

\textbf{Instructions}:

\begin{itemize}
\tightlist
\item
  Based on the results which you reported above, try and answer the
  research question: \emph{Do participants show a preference for
  ``know'' vs ``believe'' depending on their gender and the scenario
  they are in?}.
\item
  To give you an example of what we are looking for, here is a short
  paragraph on interpreting the results from the Week 8 lecture on
  morphological parsing.
\end{itemize}

\begin{quote}
As suggested by the 95\% CrI of the interaction term (in log-odds
{[}-1.72, 0.42{]}) for constituent relation type and right-branching
pairs, there is quite a lot of uncertainty in relation to the difference
in probability of correct response in unrelated vs constituent in
right-branching pairs, since the interval covers both negative and
positive values. Moreover, the conditional posterior probabilities of
unrelated and right-branching on the one hand and constituent and right
branching on the other are very similar, as can be seen in the plot (and
as suggested by the respective 95\% CrIs: 90-97\% vs 91-97\%
respectively).
\end{quote}

\end{document}
